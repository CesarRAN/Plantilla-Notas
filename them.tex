% Sistem
% The following boxes are provided:
%   Definition:     \defn 
%   Theorem:        \thm 
%   Lemma:          \lem
%   Corollary:      \cor
%   Proposition:    \prop   
%   Claim:          \clm
%   Fact:           \fact
%   Proof:          \pf
%   Example:        \ex
%   Remark:         \rmk (sentence), \rmkb (block)
% Suffix
%   r:              Allow Theorem/Definition to be referenced, e.g. thmr
%   p:              Add a short proof block for Lemma, Corollary, Proposition or Claim, e.g. lemp
%                   For theorems, use \pf for proof blocks

\usepackage{xcolor}

%-------		Un solo color		-------	

\definecolor{defscol}{HTML}{A6EB75} % Definiciones
\definecolor{asumscol}{HTML}{A6EB75}% Suposiciones

\definecolor{rmkscol}{HTML}{A6EB75} % Observaciones
\definecolor{exmscol}{HTML}{A6EB75} % Ejemplo

\definecolor{lemscol}{HTML}{A6EB75} % Lemas
\definecolor{thmscol}{HTML}{A6EB75} % Teoremas
\definecolor{prpscol}{HTML}{A6EB75} % Proposicion
\definecolor{corscol}{HTML}{A6EB75} % Corolario

\definecolor{clmscol}{HTML}{A6EB75} % Afirmacion
\definecolor{facscol}{HTML}{A6EB75} % Hecho


%-------		Varios colores		-------	
%\definecolor{defscol}{HTML}{A6EB75} % Definiciones
%\definecolor{asumscol}{HTML}{ecd8d7}% Suposiciones

%\definecolor{rmkscol}{HTML}{313160} % Observaciones
%\definecolor{exmscol}{HTML}{e04b52} % Ejemplo

%\definecolor{lemscol}{HTML}{75EB9D} % Lemas
%\definecolor{thmscol}{HTML}{75EB75} % Teoremas
%\definecolor{prpscol}{HTML}{A4EBA4} % Proposicion
%\definecolor{corscol}{HTML}{D3EBD3} % Corolario

%\definecolor{clmscol}{HTML}{D8EB75} % Afirmacion
%\definecolor{facscol}{HTML}{28a8a1} % Hecho


% ============================
% Definición
% ============================
%-------		Configuraciones		-------	
\newtcbtheorem[number within=chapter]{definicion}{Definición}
{
	enhanced,
	frame hidden,
	titlerule=0mm,
	toptitle=1mm,
	bottomtitle=1mm,
	fonttitle=\bfseries\large,
	coltitle=black,
	colbacktitle=defscol!60!,
	colback=defscol!15!white,
}{def}
%-------		Definición normal		-------	
\NewDocumentCommand{\defn}{m+m}{
	\begin{definicion}{#1}{}
		#2
	\end{definicion}
}

%-------		Definición con referencia		-------	
\NewDocumentCommand{\defnr}{mm+m}{
	\begin{definicion}{#1}{#2}
		#3
	\end{definicion}
}

% ============================
% Suposición
% ============================
%-------		Configuraciones		-------	
\newtcbtheorem[use counter from=definicion]{suposicion}{Suposición}
{
	enhanced,
	frame hidden,
	titlerule=0mm,
	toptitle=1mm,
	bottomtitle=1mm,
	fonttitle=\bfseries\large,
	coltitle=black,
	colbacktitle=defscol!60!,
	colback=defscol!10!white,
}{sup}

%-------		Suposición normal		-------	
\NewDocumentCommand{\supos}{m+m}{
	\begin{suposicion}{#1}{}
		#2
	\end{suposicion}
}
%-------		Suposición con referencia		-------	
\NewDocumentCommand{\suposr}{mm+m}{
	\begin{suposicion}{#1}{#2}
		#3
	\end{suposicion}
}

% ============================
% Teorema
% ============================

\newtcbtheorem[number within=chapter]{teorema}{Teorema}
{
	enhanced,
	frame hidden,
	titlerule=0mm,
	toptitle=1mm,
	bottomtitle=1mm,
	fonttitle=\bfseries\large,
	coltitle=black,
	colbacktitle=defscol!60!,
	colback=defscol!15!white,
}{teo}

\NewDocumentCommand{\teo}{m+m}{
	\begin{teorema}{#1}{}
		#2
	\end{teorema}
}

\NewDocumentCommand{\teor}{mm+m}{
	\begin{teorema}{#1}{#2}
		#3
	\end{teorema}
}

\newenvironment{teop}{
	{\noindent{\it \textbf{Demostración.}}}
	\tcolorbox[blanker,breakable,left=5mm,parbox=false,
	before upper={\parindent15pt},
	after skip=10pt,
	borderline west={1mm}{0pt}{thmscol!40!white}]
}{
	\textcolor{thmscol!40!white}{\hbox{}\nobreak\hfill$\blacksquare$} 
	\endtcolorbox
}

% ============================
% Lema
% ============================

\newtcbtheorem[number within=chapter]{lema}{Lema}
{
	enhanced,
	frame hidden,
	titlerule=0mm,
	toptitle=1mm,
	bottomtitle=1mm,
	fonttitle=\bfseries\large,
	coltitle=black,
	colbacktitle=defscol!60!,
	colback=defscol!15!white,
}{lem}

\NewDocumentCommand{\lem}{m+m}{
	\begin{lema}{#1}{}
		#2
	\end{lema}
}

\NewDocumentCommand{\lemr}{mm+m}{
	\begin{lema}{#1}{#2}
		#3
	\end{lema}
}

\newenvironment{lemp}{
	{\noindent{\it \textbf{Demostración.}}}
	\tcolorbox[blanker,breakable,left=5mm,parbox=false,
	before upper={\parindent15pt},
	after skip=10pt,
	borderline west={1mm}{0pt}{lemscol!40!white}]
}{
	\textcolor{lemscol!40!white}{\hbox{}\nobreak\hfill$\blacksquare$} 
	\endtcolorbox
}



% ============================
% Corolario
% ============================

\newtcbtheorem[number within=chapter]{corolario}{Corolario}
{
	enhanced,
	frame hidden,
	titlerule=0mm,
	toptitle=1mm,
	bottomtitle=1mm,
	fonttitle=\bfseries\large,
	coltitle=black,
	colbacktitle=defscol!60!,
	colback=defscol!15!white,
}{cor}

\NewDocumentCommand{\cor}{m+m}{
	\begin{corolario}{#1}{}
		#2
	\end{corolario}
}

\NewDocumentCommand{\corr}{mm+m}{
	\begin{corolario}{#1}{#2}
		#3
	\end{corolario}
}


\newenvironment{corp}{
	{\noindent{\it \textbf{Demostración.}}}
	\tcolorbox[blanker,breakable,left=5mm,parbox=false,
	before upper={\parindent15pt},
	after skip=10pt,
	borderline west={1mm}{0pt}{corscol!40!white}]
}{
	\textcolor{corscol!40!white}{\hbox{}\nobreak\hfill$\blacksquare$} 
	\endtcolorbox
}
% ============================
% Proposicion
% ============================

\newtcbtheorem[number within=chapter]{proposicion}{Proposición}
{
	enhanced,
	frame hidden,
	titlerule=0mm,
	toptitle=1mm,
	bottomtitle=1mm,
	fonttitle=\bfseries\large,
	coltitle=black,
	colbacktitle=defscol!60!,
	colback=defscol!15!white,
}{prop}

\NewDocumentCommand{\prop}{m+m}{
	\begin{proposicion}{#1}{}
		#2
	\end{proposicion}
}

\NewDocumentCommand{\propr}{mm+m}{
	\begin{proposicion}{#1}{#2}
		#3
	\end{proposicion}
}

\newenvironment{propp}{
	{\noindent{\it \textbf{Demostración.}}}
	\tcolorbox[blanker,breakable,left=5mm,parbox=false,
	before upper={\parindent15pt},
	after skip=10pt,
	borderline west={1mm}{0pt}{prpscol!40!white}]
}{
	\textcolor{prpscol!40!white}{\hbox{}\nobreak\hfill$\blacksquare$} 
	\endtcolorbox
}



% ============================
% Afirmacion
% ============================

\newtcbtheorem[use counter from=definicion]{afirmacion}{Afirmación}
{
	enhanced,
	frame hidden,
	titlerule=0mm,
	toptitle=1mm,
	bottomtitle=1mm,
	fonttitle=\bfseries\large,
	coltitle=black,
	colbacktitle=defscol!60!,
	colback=defscol!10!white,
}{af}

\NewDocumentCommand{\af}{m+m}{
	\begin{afirmacion*}{#1}{}
		#2
	\end{afirmacion*}
}

\NewDocumentCommand{\afr}{mm+m}{
	\begin{afirmacion}{#1}{#2}
		#3
	\end{afirmacion}
}

\newenvironment{afp}{
	{\noindent{\it \textbf{Demostración.}}}
	\tcolorbox[blanker,breakable,left=5mm,parbox=false,
	before upper={\parindent15pt},
	after skip=10pt,
	borderline west={1mm}{0pt}{clmscol!40!white}]
}{
	\textcolor{clmscol!40!white}{\hbox{}\nobreak\hfill$\blacksquare$} 
	\endtcolorbox
}


% ============================
% Hecho
% ============================

\newtcbtheorem[number within=chapter]{hecho}{Hecho}
{
	enhanced,
	frame hidden,
	titlerule=0mm,
	toptitle=1mm,
	bottomtitle=1mm,
	fonttitle=\bfseries\large,
	coltitle=black,
	colbacktitle=defscol!60!,
	colback=defscol!5!white,
}{hecho}

\NewDocumentCommand{\he}{mm+m}{
	\begin{hecho}{#1}{#2}
		#3
	\end{hecho}
}

% ============================
% Proof
% ============================

\NewDocumentCommand{\pf}{+m}{
	\begin{proof}
		[\noindent\textbf{Dem.}]
		#1
	\end{proof}
}

% ============================
% Ejemplo
% ============================

\newtcbtheorem[number within=chapter]{ejemplo}{Ejemplo}
{
	enhanced,
	frame hidden,
	titlerule=0mm,
	toptitle=1mm,
	bottomtitle=1mm,
	fonttitle=\bfseries\large,
	coltitle=black,
	colbacktitle=defscol!60!,
	colback=defscol!15!white,
}{ej}

\NewDocumentCommand{\ej}{m+m}{
	\begin{ejemplo*}{#1}{}
		#2
	\end{ejemplo*}
}

\NewDocumentCommand{\ejr}{mm+m}{
	\begin{ejemplo}{#1}{#2}
		#3
	\end{ejemplo}
}

\newenvironment{ejp}{
	{\noindent{\it \textbf{Demostración.}}}
	\tcolorbox[blanker,breakable,left=5mm,parbox=false,
	before upper={\parindent15pt},
	after skip=10pt,
	borderline west={1mm}{0pt}{exmscol!40!white}]
}{
	\textcolor{exmscol!40!white}{\hbox{}\nobreak\hfill$\blacksquare$} 
	\endtcolorbox
}


% ============================
% Observacion
% ============================
\newtcbtheorem[number within=chapter]{observacion}{Observación}
{
	enhanced,
	frame hidden,
	titlerule=0mm,
	toptitle=1mm,
	bottomtitle=1mm,
	fonttitle=\bfseries\large,
	coltitle=black,
	colbacktitle=defscol!60!,
	colback=defscol!15!white,
}{ob}


\NewDocumentCommand{\ob}{m+m}{
	\begin{observacion*}{#1}{}
		#2
	\end{observacion*}
}

\NewDocumentCommand{\obr}{mm+m}{
	\begin{observacion}{#1}{#2}
		#3
	\end{observacion}
}

\newenvironment{obp}{
	{\noindent{\it \textbf{Demostración.}}}
	\tcolorbox[blanker,breakable,left=5mm,parbox=false,
	before upper={\parindent15pt},
	after skip=10pt,
	borderline west={1mm}{0pt}{rmkscol!40!white}]
}{
	\textcolor{rmkscol!40!white}{\hbox{}\nobreak\hfill$\blacksquare$} 
	\endtcolorbox
}


